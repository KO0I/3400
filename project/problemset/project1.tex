\documentclass{./cls/hw}
\usepackage{graphicx,mathtools,bm}
\title{Project: Problem Set}
\course{ECEN 3400}
\author{$\boxed{\text{Patrick Harrington}}$}

% Change the subsections to (a), (b), (c), etc.
\renewcommand\thesubsection{(\alph{subsection})}

\begin{document}
\maketitle
\section{} 
A construction company looking to save some money buys an
electromagnetic crane from the ``Qo'noS Construction Company''. However, when the electromagnet arrives, the
instruction manual is in a language no one can read! However, with a bit of
cleverness on the part of the construction crew, the following information is
ascertained about the electromagnet:

\begin{itemize}
  \item The voltage supplied is 220V DC
  \item The current can be adjusted from 0-50A
  \item The ${B(H}$ dependence in the crane is 
\end{itemize}

From this information, determine the strength of the $\vec{B}$ field of the
magnet in terms of the current.

\section{}
What is the maximum energy that can be stored in the magnetic field of the
electromagnetic crane? 

\subsection{Assume that the $B(H)$ dependence is linear}
\subsection{Assume that the $B(H)$ dependence is equal to}

\section{}
Now understanding the nature of the magnet, the construction workers bundle 10
rebar rods each 10m in length. Each rebar rod has a diameter of 36mm. Given
the shape of the magnet, the workers bundle the rebar into rectangles.

\subsection{How heavy is the rebar bundle?}
\subsection{What is the $\vec{H}$ field needed to lift the rebar?}
\subsection{Assuming that the rebar bundle has a linear magnetization curve,
  how much current will the operator need to use in order to lift one rebar
  bundle? \emph{Hint: Use Ampere's Law}}

\section{}
Once the rebar hauling is completed, the workers decide to use it to move
other items on the construction site. One such item is a ring-shaped piece of
copper pipe with a diameter that is 1m. Unfortunately, the construction
workers forgot that copper is diamagnetic, and assume that the metal ring is
simply stuck. The  confused operator of the crane moves the magnet only inches
away from the ring, such that the $\vec{B}$ field through it is effectively
uniform. When this has no effect, he fiddles with various knobs that have an unknown purporse, and asks his friend to ``free'' the pipe while he tries to get the crane to work
``better''. One such button  the frustrated operator presses initializes a
sinusoidal oscillation in the supplied voltage of 60Hz, with the RMS value
being equal to the DC value of the crane in DC mode (220V). The operator's
friend is not wearing any gloves, and has exceptionally sweaty hands.

When the operator's coworker grabs the metal ring, he grabs it such that there is 1/3 of the circumference of the ring between his two hands
\subsection{ How much current passes through the coworker?}
\subsection{ Will the coworker live?}

\section{}
During break, many of the workers are watching TV on an old CRT set.
However, the workers have since continued watching TV instead of returning to
work, and the impatient foreman decides to get their attention by using the
electromagnetic crane.

With the magnetic unit positioned a very short height over the set, the foreman
fires up the electromagnet as seen in the figure below. Assuming that the TV
is smaller than the magnet (and that the $\vec{B}$ field is approximately
uniform)

\subsection{Describe and/or sketch what the workers see on the screen once the
magnet has stabilized.}
\subsection{The foreman is determined to try and destroy the television. Even
with the magnet on top of the set and the current at maximum, can he do it?
What current value will destroy the set?}
\subsection{If the foreman engages the oscillatory mode in a fit of rage, what
will happen?}



\end{document}

